% timv's frequently used tex configuration.

%\documentclass[10pt]{article}
\usepackage[usenames,dvipsnames]{color} %used for font color
\usepackage{amssymb} %maths
\usepackage{amsmath} %maths
%\usepackage[utf8]{inputenc} %useful to type directly diacritic characters
\usepackage[makeroom]{cancel} % For cancel

\setlength\parindent{0pt}   % no paragraph indentation
\usepackage{fullpage}
\setlength{\parskip}{.4cm plus4mm minus3mm}

%\usepackage{lmodern}

% fontenc is oriented to output, that is, what fonts to use for printing characters.
%\usepackage[T1]{fontenc}

% inputenc allows the user to input accented characters directly from the keyboard;
%\usepackage[utf8]{inputenc}
%\usepackage[latin1]{inputenc}

\usepackage{fancybox}
\usepackage{pgf}
\usepackage{tikz}
\usetikzlibrary{arrows,automata}

\usepackage{framed,color}
\definecolor{shadecolor}{rgb}{1,0.8,0.3}

\usepackage{multirow}
\usepackage{url}
\usepackage{graphicx}

\usepackage{latexsym}
\usepackage{amsfonts}
\usepackage{amsmath}
\usepackage{amsthm}
\usepackage{amssymb}
\usepackage{amsbsy}

\usepackage{verbatim}
\usepackage{xspace}
\usepackage{url}
\usepackage{algorithm2e}


\usepackage{color}
\usepackage{xcolor}
\definecolor{darkgrey}{rgb}{0.2,0.2,0.2}
\definecolor{grey}{rgb}{0.9,0.9,0.9}
\definecolor{darkblue}{rgb}{0.0,0.0,0.5}
\definecolor{darkpurple}{rgb}{0.4,0.0,0.4}
\definecolor{darkred}{rgb}{0.5,0.0,0.0}
\definecolor{darkorange}{rgb}{0.5,0.45,0.4}
\definecolor{darkgreen}{rgb}{0.0,0.5,0.0}
\definecolor{darkergreen}{rgb}{0.0,0.4,0.0}
\definecolor{lightblue}{rgb}{0.8,0.8,1.0}
\definecolor{lightgreen}{rgb}{0.8,1.0,0.8}
\definecolor{lightred}{rgb}{1.0,0.8,0.8}
\definecolor{lightyellow}{rgb}{1.0,1.0,0.8}
\definecolor{lightorange}{rgb}{1.0,0.9,0.8}
\definecolor{lightgrey}{rgb}{0.96,0.97,0.98}
\definecolor{orange}{HTML}{D95F02}   % darker /default/ orange color.

\newsavebox{\savelisting}
\newenvironment{timbox}
{\begin{lrbox}{\savelisting}
\begin{minipage}{6.5in}
\begin{flushleft}}
{\end{flushleft}
\end{minipage}
\end{lrbox}
\begin{center}
\resizebox{\columnwidth}{!}{\setlength\fboxsep{6pt}\fbox{\usebox{\savelisting}}}
\end{center}}

\usepackage{hyperref}
\hypersetup{
  colorlinks=true,
  allcolors=blue
}

\usepackage{bm}             % load after all math to give access to bold math

%\usepackage{hyperref}
\tolerance=1000
\providecommand{\alert}[1]{\textbf{#1}}


\newcommand{\Note}[3]{{\textcolor{#2}{[\textbf{#1:} #3]}}}
\newcommand{\todo}[1]{\Note{TODO}{red}{#1}}
\newcommand{\timv}[1]{\Note{timv}{magenta}{#1}}
\newcommand{\halt}{\textsc{halt}\xspace }
\newcommand{\astar}{A$^*$\xspace}
\newcommand{\softmin}{{\text{{softmin}}}}
\newcommand{\softmax}{{\text{{softmax}}}}


\newcommand{\Var}{\mathrm{Var}}
\newcommand{\Cov}{\mathrm{Cov}}

\newcommand{\loss}[1]{ \mathcal{L}\left( #1 \right) }
\newcommand{\indicator}[1]{ \textbf{1}\left[ #1 \right] }

\renewcommand{\|}{\textrm{~}\arrowvert\textrm{~}}

\newcommand{\R}{\ensuremath{\mathbb{R}}}

\newcommand{\Expect}[2]{\mathop{\mathbb{E}}_{#1}\left[#2\right]}
\newcommand{\Ebb}{\mathbb{E}}

\newcommand{\ExpectA}[0]{\mathbb{E}}
\newcommand{\ExpectB}[1]{\mathbb{E}\!\left[ #1 \right]}
\newcommand{\ExpectC}[2]{\mathbb{E}_{#1}\!\left[ #2 \right]}

%\newcommand{\Ebb}[1]{\mathbb{E} \left[ #1 \right] }
%\DeclareMathOperator*{\Eee}{\mathbb{E}}
%\newcommand{\E}[2]{  \mathop{\mathbb{E}}_{#1} \Big[ #2 \Big] }
%\newcommand{\EE}[2]{ \mathop{\mathbb{E}}_{#1} \left[ #2 \right] }

\newcommand{\tcdot}{\! \cdot\! }

\DeclareMathOperator*{\argmax}{argmax}
\DeclareMathOperator*{\argmin}{argmin}

\newcommand{\g}{\,|\,}
\newcommand{\teq}{\!=\!}
\newcommand{\tcdot}{\! \cdot\! }

\newcommand{\bphi}{\boldsymbol{\phi}}
\newcommand{\btheta}{\boldsymbol{\theta}}


\newcommand{\grad}[2]{\nabla_{\! #1}\! \left[ #2 \right]}

\newcommand{\gradtheta}[1]{\grad{\theta}{ #1 }}
\newcommand{\gradx}[1]{\grad{x}{ #1 }}
\newcommand{\sqnorm}[1]{|| #1 ||^2}
\newcommand{\norm}[1]{|| #1 ||}

\newcommand{\pair}[2]{\left<#1, #2\right>}
\newcommand{\tuple}[1]{\left<#1\right>}

\newcommand{\iid}{\text{i.i.d.}\xspace}

\newcommand{\bigo}{\mathcal{O}}
\newcommand{\wrt}{\text{w.r.t.}\xspace}
\newcommand{\suchthat}{\text{s.t.}\xspace}
\newcommand{\iid}{\text{i.i.d. }}
\newcommand{\bigo}{\mathcal{O}}
\newcommand{\notimplies}{%
  \mathrel{{\ooalign{\hidewidth$\not\phantom{=}$\hidewidth\cr$\implies$}}}}
